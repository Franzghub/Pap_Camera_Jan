%!TEX root = ../ArticleCalib_main.tex

%%%%%%% INTRODUCTION article calib
\section{Introduction}

In the recent years, increasingly sensitives detectors are used to detect localised sources, like small objects or single molecules, but those low noise point detectors are unsuitable for extended sources or diluted emitters. 
Infra-noise light fluxes produced by an extended surface needs larges light detectors ; however increasing the surface of a detector increases its noise, and its noise's variance.

Among all detectors, there are many advantages of EMCCDs, such as sub-electron read-out noise and low dark current, for low light applications. Nowadays their low areal noise and high detection capacities permit to reach single photon detection ; however, detecting a few photon per hour stays a challenge, more so when emitted by a large surface. To achieve a sufficient signal over noise ratio (SNR) for significative detection, one must carefully consider and explore the operating mode of the camera, to reach a detection's optimum for a given light flux.

In our work, we address the detection a light flux intensity (1 photon/sec/cm$^2$) given by a large area, as compared to the sensor's surface, which is lower than our detector equivalent areal noise. % combiner les deux

Concerning light flux detection, 3 situations have to be considered :
\begin{itemize}
\item constant uniform flux,
\item variable uniform flux :  statistic filtering,
\item non uniform flux : detection of an object's presence.
\end{itemize}

Here, we propose a statistical method, combined with an EMCCD detector, for an optimised detection of constant, uniform, and low light fluxes emitted by dim extended objects.
We present the camera we choose and its use's mode,  describe the statistical model of the detector, and show the experimental pixel by pixel characterisation and behaviour of the detector. 
Then we build a model to simulate the detector's response to a low, constant, uniform flux, and to extract the SNR as well as two characteristics exposure times, in order to use the detector optimally.






