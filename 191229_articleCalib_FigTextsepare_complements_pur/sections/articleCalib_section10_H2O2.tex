%!TEX root = ../ArticleCalib_main.tex

%%%%%% H2O2  article calib
\section{H$_2$O$_2$ chemiluminescence in water}

\subsection{methods}
Measurements are done in a black box with a controlled temperature at 23\pm 0.5\°c and with an air flow at 50mL/min.
Solutions are placed in an integrative sphere which opening is conjugated to the camera with a lense by maximizing the light collection. All experiments show only the image part corresponding to the opening of the integrative sphere. Controls were always made to assure that the light came from inside the integrative sphere (data not shown) and not from an environment or camera noise fluctuation)
All experiments are done without opening the black box in between. Solutes are injected and removed manually remotely with tubes and seringes.
Thorough washing with milliQ water is done between concentration and conditions, and concentration are injected increasingly to avoid bias.
For all experiments, milliQ water filtered at 200nm was used as a solvent and kept in the dark. 10mL was always used during experiments as control or dilution volume. 
H$_2$O$_2$ (sigma ****) solutions were prepared fresh the same day using the same water and diluted from a 30\% stock solution aged less than a month.
Human hemoglobin was used as a catalyst and prepared at 0.5mg/ml, 1mL of this solution was injected in 10mL of water with the indicated sequence. 
All experiments were repeated at least twice, on a different camera and integrative sphere. The results present the experiment done on one camera and integrative sphere.
Data are analysed with python and smoothed using a median filter.

\subsection{Results}
In presence of catalyst (\subref{fig:H2O2:A}, \subref{fig:H2O2:B}) , the injection of H$_2$O$_2$ denote a clear peak of light, concentration dependant, and independant of the temperature variation (data not shown). Indeed the temperature variation is \pm ******* and the NETD is inferior at this time of exposure.

**calcule moles ?

Water doesn't present a detectable delayed luminescence for those times of exposure, however for the highest concentration we could guess to a luminescence beginning in the absence of catalyst.

We tested several concentration and a reproducible and significative light emission was detected with injection of 1mL of 3\% H$_2$O$_2$ in 10mL of water and persistent for hours (\subref{fig:H2O2:C}).
The washing of the pshere reverse this luminescence.

The testing of lower concentrations at the time of exposure of maximal density reproduce the previous result (\subref{fig:H2O2:D}) but seems to give more ambiguous results for lower concentrations.


Those results were described before (publication UWL H2O2), however never with the controls (imaging of the emission, temperature and quantification posiibilities) and precision that we demonstrate.
We didn't test the spectrum of the emission, however the litterature point toward a singlet oxygen light emission. As our camera spectrum is limited to the visible, if the light emission indeed comes from the signlet oxygen, it should be from its dimolar form emitting at ***** mostly.
Further experiments are to be made to confirm or inform this hypothesis.
