%!TEX root = ../ArticleCalib_main.tex

%%%%%% FLUCTUATING OR HETEREGONEOUS FLUX MEASUREMENT ? article calib
\section{Practically}

We address here different practical issues or questions about fast SNR simulation, cosmic rays, and and thermal radiation detection

%% Fast SNR simulation
\subsection{Fast SNR simulation}

Taking into account the 513$\times$512 pixels into a sum of Bernoulli random variable with their own parameter is time consuming, and for the single pixel characterisation, and for the simulation processing. 
If a subtle knowledge of the camera pixels is not necessary, estimating the SNR by a model is swiftly done by considering the detector homogeneous, with the same CIC and Id parameter for all pixels. The parameters are extracted by doing a simple linear regression ($\tilde{f_{counts}} = \tilde{f_{CIC}} + \tilde{f_{Id}}*\tau$) out of saturation on the global detector response on complete dark conditions (see \ref{fig:PixByPix:A}).\par
The difference between both homogeneous model (all pixels's noises identical), and the heterogeneous model (all pixels's noises singular), is the variance between pixels on one frame, $\sigma^2_{ij}$. Surprisingly enough, this $\sigma^2_{ij}$ is subtracted to the N1 variance in the heterogeneous model case, which leads to a better SNR (\ref{fig:SNRTau:HeteroHomo}). 
It is easily demonstrated by :   \par

\textcolor{blue}{demonstration de la difference}.\par
\medskip
However, the difference is small enough (from 1 to 7\%) for neglecting it for the sake of simplicity. 
%
	%\input{figures/articleCalib_fig4supp_model_HeteroHomo}
	
%% Cosmic rays
\subsection{Cosmic rays}

Cosmic rays (CR) are particles of different kinds, carrying a wide range of energies. They arrive from the cosmos and trigger multipixel response patterns on the detector (see \ref{fig:CR:A}). 
Because these pixel clusters patterns bear some reproducible features (a cluster of connected pixels followed by a "horizontal tail"), we wrote a program to detect a given number of connected pixels, this cut off number of connected pixels being determined by a negligible probability ($<10^{-6}$) to happen randomly for a given time of exposure. The frequency of events (ie. cluster of connected pixels), as function of the number of pixels in the cluster, was determined by a Monte Carlo simulation, on an arbitrary large enough detector to achieve sufficient statistic, using our model to produce the noise pattern, for different time of exposure. This program create a detector equivalent logical matrix attributing "ones" to the pixels belonging to a cluster's size bigger than the determined cut off. After indexing the concerned pixels, we substitute them by random values drawn from the appropriate Bernoulli distribution.
Because the detection of these clusters relies on the recognition of connected pixels, it becomes more difficult to tell them from clusters generated by signal photons when the density of positive pixels is too high, e.g. when the time of exposure exceeds 2000 seconds. 
However, we could assess the contribution of cosmic rays and we found that, in photon counting mode, it amounts to $4,1.10^{-6}$ electrons/sec/pixels, that is $1/20$ of the dark current (figures \ref{fig:CR:B} and  \ref{fig:CR:C}.) \par

Importantly, this equivalence with an effective photon rate determined in binary mode (photon counting) is physically meaningless, because it does not reflect the actual strength of cosmic rays which saturate analog detectors.
Digital treatment makes these high energy perturbations look like regular visible photons, and this can be considered as an additional advantage of using a binary mode to detect low light fluxes. \par
We consider that the impact of the cosmic ray on our model is negligible, as far as the concerned time of exposure are small enough that their removal doesn't impact significantly the statistic of the wide detector response.\par
%
	%\input{figures/articleCalib_Fig5_CR}


%% BBR
\subsection{Black Body Radiation}

Given the SNR model a rightful question emerged : what is the sensitivity of our camera to thermal radiation ? Indeed, the signal of interest being so weak, it is fundamental to be able to distinguish it from other kind of radiation and fluctuation linked to temperature. This is even more crucial given that we aim for biological sample that are kept at higher temperature (37 celsius degrees) than room temperature.\par
%During extensive measures trying to uncover the origin of N1 outliers, we noticed significant fluctuations happening in the background noise producing a double stochasticity effect.
%We went further by measuring directly the thermal flux with the help of a Black Body and we realised that we were sensitive to changes of temperature (see figure \ref{fig:BBR:exp_100s_600s}). 

Therefore, we modelled the sensitivity of our detector to temperature changes, considering its spectral sensitivity (see figure \ref{fig:BBR:NuvuQESpectra}), and the black body radiation spectra (\ref{fig:BBRtheo1:A}.). \par
Our detector is indeed sensitive to temperature, according to the model, and in a much more drastic way that we expected.  Given a time of exposure of 600 sec, the SNR is 10 and 100 (according to our SNR model (see figure \ref{fig:SNRTau:B}) for the thermal flux given at a temperature of  23-25c and 37-38c respectively. The saturation of the detector happens for 58\°c ( see figure \ref{fig:BBRtheo1:C}). The Noise Equivalent Temperature Difference (NETD) for a thermal flux given at temperatures 20c, 26c, 32c and 37c is respectively 2, 1, 0.5, 0.2 (figure \ref{fig:BBRtheo2}).\par
It means that for a sample maintained at 37c, the camera would be sensitive to  a 0.2c change of temperature. It shows that the measured level of our background for a given $\tau$ is actually temperature dependent.  \par
This very recent result raises new challenges for temperature and environment control that are for now unprecedented and was never considered in this field.\par

%	
	%\input{figures/articleCalib_fig6_BBRtheocalibration}


