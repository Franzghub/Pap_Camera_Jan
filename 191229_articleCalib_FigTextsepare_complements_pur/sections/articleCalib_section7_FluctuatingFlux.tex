%!TEX root = ../ArticleCalib_main.tex

%%%%%% FLUCTUATING OR HETEREGONEOUS FLUX MEASUREMENT ? article calib
\section{Fluctuating light flux and heterogeneous emission}

The measure of a fluctuating light flux implies a statistic filtration to detect a double stochasticity.
In other words, when sampling the acquisition, do the fluctuations exceed the expected stochastic fluctuations of the observable ?
If the exposure time is larger than the fluctuation period, we won't determine if there are signal variations, whereas if the exposure time is smaller than the fluctuation period, but the signal is too weak to achieve a sufficient SNR per frame, or by repeating frames, we won't determine if there is a signal at all. \par
Hence, there is a time resolution limit for a given light flux, depending on the minimal time of exposure to be repeated or to be met to have a sufficient SNR, compared to the variation period and amplitude of the signal.\par
\medskip


To image an object emitting an heterogeneous light flux, we have to address the question in part \ref{sec:SinglePixCaract} :  is a pattern of pixels consistent with the random effect of a perfectly uniform optical field, or is it produced by a contrasted object ? Knowing perfectly the background of each pixels for a given exposure time permits to do tests of spatial randomness to detect an non uniform light field on the detector. A pixel by pixel study is less sensible than a hole detector detection as it will rely uniquely on the number of repetitions of frames to make a significative difference in between pixels. \par
If we know the image position on the captor, and if the image light flux is uniform, the strategy can be different : to increase the detection sensibility, we can then sum all pixels of the background area and the lit area, as we did for the entire detector, and compare and quantify both observables.\par
Combining methods as per fit is one's strategic choice to make.






